%%%%%%%%%%%%%%%%%%%%%%%%%%%%%%%%%%%%%%%%%
% FRI Data Science_report LaTeX Template
% Version 1.0 (28/1/2020)
% 
% Jure Demšar (jure.demsar@fri.uni-lj.si)
%
% Based on MicromouseSymp article template by:
% Mathias Legrand (legrand.mathias@gmail.com) 
% With extensive modifications by:
% Antonio Valente (antonio.luis.valente@gmail.com)
%
% License:
% CC BY-NC-SA 3.0 (http://creativecommons.org/licenses/by-nc-sa/3.0/)
%
%%%%%%%%%%%%%%%%%%%%%%%%%%%%%%%%%%%%%%%%%


%----------------------------------------------------------------------------------------
%	PACKAGES AND OTHER DOCUMENT CONFIGURATIONS
%----------------------------------------------------------------------------------------
\documentclass[fleqn,moreauthors,10pt]{ds_report}
\usepackage[english]{babel}

\graphicspath{{fig/}}




%----------------------------------------------------------------------------------------
%	ARTICLE INFORMATION
%----------------------------------------------------------------------------------------

% Header
\JournalInfo{FRI Natural language processing course 2021}

% Interim or final report
\Archive{Project report} 
%\Archive{Final report} 

% Article title
\PaperTitle{Automatic generation of Slovenian traffic news for RTV Slovenija} 

% Authors (student competitors) and their info
\Authors{Aljaž Justin, Edin Ćehić and Lea Briški}

% Advisors
\affiliation{\textit{Advisors: Slavko Žitnik}}

% Keywords
\Keywords{LLM, Traffic}
\newcommand{\keywordname}{Keywords}


%----------------------------------------------------------------------------------------
%	ABSTRACT
%----------------------------------------------------------------------------------------

\Abstract{
% The abstract goes here. The abstract goes here. The abstract goes here. The abstract goes here. The abstract goes here. The abstract goes here. The abstract goes here. The abstract goes here. The abstract goes here. The abstract goes here. The abstract goes here. The abstract goes here. The abstract goes here. The abstract goes here. The abstract goes here. The abstract goes here. The abstract goes here. The abstract goes here. The abstract goes here. The abstract goes here. The abstract goes here. The abstract goes here. The abstract goes here. The abstract goes here. The abstract goes here. The abstract goes here.
}

%----------------------------------------------------------------------------------------

\begin{document}

% Makes all text pages the same height
\flushbottom 

% Print the title and abstract box
\maketitle 

% Removes page numbering from the first page
\thispagestyle{empty} 

%----------------------------------------------------------------------------------------
%	ARTICLE CONTENTS
%----------------------------------------------------------------------------------------

\section*{Introduction}
	Our initial goal is to leverage an open-source large language model (LLM) capable of generating text in Slovenian to automate the process of traffic report generation for RTV SLO. By doing so, we aim to streamline and enhance the efficiency of producing structured reports while ensuring consistency in formatting and adherence to predefined guidelines.

    To achieve this, we plan to train the LLM to learn the specific formatting and stylistic requirements from existing data. This will allow the model to generate reports that align with the required standards without the need for extensive manual editing. Since using a large-scale model can be resource-intensive, we also intend to fine-tune a smaller, more efficient LLM tailored to our specific use case. This smaller model would retain the necessary capabilities while reducing computational costs, making it more practical for real-world deployment.
    
    If time permits, we aim to extend the project by integrating automatic translation features for languages commonly spoken in Slovenia, such as English, German, and Italian. This would make the generated reports more accessible to a wider audience. As a potential final step, we may also explore the implementation of text-to-speech functionality, particularly for English, to further enhance accessibility and usability.
    
    Through this approach, we seek to develop a scalable and efficient solution for automated report generation that meets the needs of Slovenian-speaking users while incorporating multilingual and accessibility features as an added value.
    

%------------------------------------------------

\section*{Related works}

This work focuses on the automatic generation of traffic reports from traffic data. This intersects with several areas of research, including data-to-text generation, natural language processing for traffic information, and the application of large language models (LLMs).

\textbf{Data-to-Text Generation:} The task of automatically generating textual descriptions from structured data has been a long-standing research area \cite{Data2Text}. Our work contributes to this field by focusing on the specific domain of traffic data, aiming to produce informative and concise reports. Recent advancements in LLMs have significantly impacted data-to-text generation, offering new possibilities for creating more natural and contextually relevant outputs \cite{LlamaIndex2025}.

\textbf{Traffic Information Processing with NLP:} Several works have explored the use of NLP techniques for processing and extracting information from traffic-related sources. For instance, \cite{articleRTTRS} investigates empowering real-time traffic reporting systems using NLP-processed social media data. Our work complements this by focusing on generating reports directly from structured traffic data. The automation of traffic incident management using loop data has also been explored recently \cite{cercola2025automatinglooptrafficincident}.

\textbf{Large Language Models for Text Generation and Understanding:} The rise of large language models has opened new avenues for various natural language processing tasks, including text generation \cite{vreš2024generativemodellessresourcedlanguage, zhu2024multilingualmachinetranslationlarge, pelofske2024automatedmultilanguageenglishmachine, peng2024automaticnewsgenerationfactchecking}. Techniques such as prompt engineering \cite{white2023promptpatterncatalogenhance} and fine-tuning \cite{j2024finetuningllmenterprise, j2024finetuningllmenterprise} are crucial for adapting these models to specific domains and tasks. Furthermore, the ability of LLMs to understand and utilize document layout for enhanced performance has been investigated \cite{10.1007/978-3-031-70546-5_9}. While our work primarily focuses on generating text from structured data, these advancements in LLM capabilities are highly relevant.

\textbf{Evolution of Collective Behavior and Linguistic Systems:} Although seemingly distant, the foundational work on the evolution of collective behavior using linguistic fuzzy rule-based systems \cite{Demsar2017LinguisticEvolution} and the promotion of parallel movement through balanced antagonistic pressures \cite{Demsar2016BalancedMixture} provide insights into the development of complex communicative behaviors in artificial systems, which can indirectly inform the design of more sophisticated traffic reporting systems in the future.
 
Article \textbf{A Prompt Pattern Catalog to Enhance Prompt Engineering with ChatGPT} explores possibilities of formatting prompts for existing LLMs, guiding them to ensure certain form and rules of the generated output. This could be used for the first part of the task, where our task would lead us to initially use prompt engineering for traffic report generation.

\section*{Methods}

% Use the Methods section to describe what you did an how you did it -- in what way did you prepare the data, what algorithms did you use, how did you test various solutions ... Provide all the required details for a reproduction of your work.

% Below are \LaTeX examples of some common elements that you will probably need when writing your report (e.g. figures, equations, lists, code examples ...).


% \subsection*{Equations}

% You can write equations inline, e.g. $\cos\pi=-1$, $E = m \cdot c^2$ and $\alpha$, or you can include them as separate objects. The Bayes’s rule is stated mathematically as:

% \begin{equation}
% 	P(A|B) = \frac{P(B|A)P(A)}{P(B)},
% 	\label{eq:bayes}
% \end{equation}

% where $A$ and $B$ are some events. You can also reference it -- the equation \ref{eq:bayes} describes the Bayes's rule.

% \subsection*{Lists}

% We can insert numbered and bullet lists:

% % the [noitemsep] option makes the list more compact
% \begin{enumerate}[noitemsep] 
% 	\item First item in the list.
% 	\item Second item in the list.
% 	\item Third item in the list.
% \end{enumerate}

% \begin{itemize}[noitemsep] 
% 	\item First item in the list.
% 	\item Second item in the list.
% 	\item Third item in the list.
% \end{itemize}

% We can use the description environment to define or describe key terms and phrases.

% \begin{description}
% 	\item[Word] What is a word?.
% 	\item[Concept] What is a concept?
% 	\item[Idea] What is an idea?
% \end{description}


% \subsection*{Random text}

% This text is inserted only to make this template look more like a proper report. Lorem ipsum dolor sit amet, consectetur adipiscing elit. Etiam blandit dictum facilisis. Lorem ipsum dolor sit amet, consectetur adipiscing elit. Interdum et malesuada fames ac ante ipsum primis in faucibus. Etiam convallis tellus velit, quis ornare ipsum aliquam id. Maecenas tempus mauris sit amet libero elementum eleifend. Nulla nunc orci, consectetur non consequat ac, consequat non nisl. Aenean vitae dui nec ex fringilla malesuada. Proin elit libero, faucibus eget neque quis, condimentum laoreet urna. Etiam at nunc quis felis pulvinar dignissim. Phasellus turpis turpis, vestibulum eget imperdiet in, molestie eget neque. Curabitur quis ante sed nunc varius dictum non quis nisl. Donec nec lobortis velit. Ut cursus, libero efficitur dictum imperdiet, odio mi fermentum dui, id vulputate metus velit sit amet risus. Nulla vel volutpat elit. Mauris ex erat, pulvinar ac accumsan sit amet, ultrices sit amet turpis.

% Phasellus in ligula nunc. Vivamus sem lorem, malesuada sed pretium quis, varius convallis lectus. Quisque in risus nec lectus lobortis gravida non a sem. Quisque et vestibulum sem, vel mollis dolor. Nullam ante ex, scelerisque ac efficitur vel, rhoncus quis lectus. Pellentesque scelerisque efficitur purus in faucibus. Maecenas vestibulum vulputate nisl sed vestibulum. Nullam varius turpis in hendrerit posuere.


% \subsection*{Figures}

% You can insert figures that span over the whole page, or over just a single column. The first one, \figurename~\ref{fig:column}, is an example of a figure that spans only across one of the two columns in the report.

% \begin{figure}[ht]\centering
% 	\includegraphics[width=\linewidth]{single_column.pdf}
% 	\caption{\textbf{A random visualization.} This is an example of a figure that spans only across one of the two columns.}
% 	\label{fig:column}
% \end{figure}

% On the other hand, \figurename~\ref{fig:whole} is an example of a figure that spans across the whole page (across both columns) of the report.

% % \begin{figure*} makes the figure take up the entire width of the page
% \begin{figure*}[ht]\centering 
% 	\includegraphics[width=\linewidth]{whole_page.pdf}
% 	\caption{\textbf{Visualization of a Bayesian hierarchical model.} This is an example of a figure that spans the whole width of the report.}
% 	\label{fig:whole}
% \end{figure*}


% \subsection*{Tables}

% Use the table environment to insert tables.

% \begin{table}[hbt]
% 	\caption{Table of grades.}
% 	\centering
% 	\begin{tabular}{l l | r}
% 		\toprule
% 		\multicolumn{2}{c}{Name} \\
% 		\cmidrule(r){1-2}
% 		First name & Last Name & Grade \\
% 		\midrule
% 		John & Doe & $7.5$ \\
% 		Jane & Doe & $10$ \\
% 		Mike & Smith & $8$ \\
% 		\bottomrule
% 	\end{tabular}
% 	\label{tab:label}
% \end{table}


% \subsection*{Code examples}

% You can also insert short code examples. You can specify them manually, or insert a whole file with code. Please avoid inserting long code snippets, advisors will have access to your repositories and can take a look at your code there. If necessary, you can use this technique to insert code (or pseudo code) of short algorithms that are crucial for the understanding of the manuscript.

% \lstset{language=Python}
% \lstset{caption={Insert code directly from a file.}}
% \lstset{label={lst:code_file}}
% \lstinputlisting[language=Python]{code/example.py}

% \lstset{language=R}
% \lstset{caption={Write the code you want to insert.}}
% \lstset{label={lst:code_direct}}
% \begin{lstlisting}
% import(dplyr)
% import(ggplot)

% ggplot(diamonds,
% 	   aes(x=carat, y=price, color=cut)) +
%   geom_point() +
%   geom_smooth()
% \end{lstlisting}

% %------------------------------------------------

% \section*{Results}

% Use the results section to present the final results of your work. Present the results in a objective and scientific fashion. Use visualisations to convey your results in a clear and efficient manner. When comparing results between various techniques use appropriate statistical methodology.

% \subsection*{More random text}

% This text is inserted only to make this template look more like a proper report. Lorem ipsum dolor sit amet, consectetur adipiscing elit. Etiam blandit dictum facilisis. Lorem ipsum dolor sit amet, consectetur adipiscing elit. Interdum et malesuada fames ac ante ipsum primis in faucibus. Etiam convallis tellus velit, quis ornare ipsum aliquam id. Maecenas tempus mauris sit amet libero elementum eleifend. Nulla nunc orci, consectetur non consequat ac, consequat non nisl. Aenean vitae dui nec ex fringilla malesuada. Proin elit libero, faucibus eget neque quis, condimentum laoreet urna. Etiam at nunc quis felis pulvinar dignissim. Phasellus turpis turpis, vestibulum eget imperdiet in, molestie eget neque. Curabitur quis ante sed nunc varius dictum non quis nisl. Donec nec lobortis velit. Ut cursus, libero efficitur dictum imperdiet, odio mi fermentum dui, id vulputate metus velit sit amet risus. Nulla vel volutpat elit. Mauris ex erat, pulvinar ac accumsan sit amet, ultrices sit amet turpis.

% Phasellus in ligula nunc. Vivamus sem lorem, malesuada sed pretium quis, varius convallis lectus. Quisque in risus nec lectus lobortis gravida non a sem. Quisque et vestibulum sem, vel mollis dolor. Nullam ante ex, scelerisque ac efficitur vel, rhoncus quis lectus. Pellentesque scelerisque efficitur purus in faucibus. Maecenas vestibulum vulputate nisl sed vestibulum. Nullam varius turpis in hendrerit posuere.

% Nulla rhoncus tortor eget ipsum commodo lacinia sit amet eu urna. Cras maximus leo mauris, ac congue eros sollicitudin ac. Integer vel erat varius, scelerisque orci eu, tristique purus. Proin id leo quis ante pharetra suscipit et non magna. Morbi in volutpat erat. Vivamus sit amet libero eu lacus pulvinar pharetra sed at felis. Vivamus non nibh a orci viverra rhoncus sit amet ullamcorper sem. Ut nec tempor dui. Aliquam convallis vitae nisi ac volutpat. Nam accumsan, erat eget faucibus commodo, ligula dui cursus nisi, at laoreet odio augue id eros. Curabitur quis tellus eget nunc ornare auctor.


% %------------------------------------------------

% \section*{Discussion}

% Use the Discussion section to objectively evaluate your work, do not just put praise on everything you did, be critical and exposes flaws and weaknesses of your solution. You can also explain what you would do differently if you would be able to start again and what upgrades could be done on the project in the future.


%------------------------------------------------

% \section*{Acknowledgments}

% Here you can thank other persons (advisors, colleagues ...) that contributed to the successful completion of your project.


%----------------------------------------------------------------------------------------
%	REFERENCE LIST
%----------------------------------------------------------------------------------------
\bibliographystyle{unsrt}
\bibliography{report}


\end{document}